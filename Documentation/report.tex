\documentclass[11pt]{article}
\usepackage{fullpage}
\usepackage{tikz}
\usepackage{graphicx}
\usepackage{hyperref}
\pagenumbering{gobble}
\hypersetup{
    colorlinks=true,
    linkcolor=blue,
    filecolor=red,
	urlcolor=magenta,
}

\begin{document}

\begin{tikzpicture}[remember picture,overlay]
	\node[anchor=north west,yshift=-20pt,xshift=60pt]%
	at (current page.north west)
	{\includegraphics[scale=0.5]{djangoVue.png}};
\end{tikzpicture}

\begin{flushright}
	\Huge{\textbf{Online Resume}} \\
	\Large{Using Django, Vue.js and MongoDB} \\
	\emph{\large{\today}} \\
\end{flushright}

\begin{flushleft}
	\large{\textbf{Prepared by: } Muhammad Bilal Khan} \\
\end{flushleft}

\vspace{-0.8cm}

\begin{center}
	{\rule{475pt}{1pt}}
\end{center}
	
\begin{flushright}
	Hosted at: \href{http://ec2-52-23-221-163.compute-1.amazonaws.com:5000/}{Amazon Web Service} \\
	Source Code: \href{'https://github.com/Bill-Klay/Online-Resume}{Git Hub} \\
	% Heroku: \\
\end{flushright}

\section{Introduction}
This report was establish for the purpose of deployment of an online website implemented using \href{https://www.djangoproject.com/}{Django} as the backend engine and \href{https://vuejs.org/}{Vue.js} as the frontend for manipulating the DOM plus \textbf{MongoDB} for the database.
The report discusses the layout of the project folder and describes the content held within.
Overviewing and bit about which file implements what and how.
The \textbf{links} stated at the start of the documents links to some of the important aspects of the task.
\textit{Amazon Web Service} holds the code made live running on a virtual machine.
% \textit{Heroku} has the django app deployed on a Python dyno. 
Please visit my \textit{Git Hub} for viewing and inspecting the complete code.
\textit{Please consult this report inplace for the read.me file; writing this in markdown is a task for another day.}

\section{File Structure}
The repository contains 2 folders mainly
\begin{enumerate}
	\item website-project
	\item Documentation
\end{enumerate}
In the order of alphabetic listing each folder has been described. 
\textbf{Blog} folder is an app created within the Django app, it holds the model implementation for Django ORM for a blogs page which has been left for future implementation. 
Like wise the \textbf{homepage} foler is app and model implementation for the landing page and deals with communicating with the \textbf{MongoDB} port running at the back. 
You'll find the database model here as well. 
It also holds the \textit{index.html} file within the \textbf{templates} folder which served for the purpose of displaying the home or the landing page of the single page application. 
\textbf{Media} is the folder for the Django ORM to store files and database structure for accessability, currently it only holds images, hence just a sinel \textbf{images} folder within. 
\textbf{Static} folder the location for Django to serve static files from. 
CSS, JavaScript, images and PDf docuemnts have been collected using the \textit{collectstatic} functionality of Django. 
\textbf{Website} is the project folder, and houses the setting, url and views file; although they would be found empty, since the website has been redirected to the homepage app of the project. 
The \textit{settings file} will provide information on the database that has been set up and some other dependencies. 
An important feature of the project is that it has been \textit{dockerized}, so just prepare \textit{docker engine} and \textit{docker compose} and fire up the \textbf{Dockerfile} to run the website regardless of the requirements. 
Speaking of requirements, \textbf{requirements.txt} is the file created from the python virtual environment and holds the python required to run the django server. \par
\textbf{Documentation} folder houses this report and couple other \LaTeX\ dependencies.

\section{Deployment}
For the purpose of hosting this single page application I've utilized my \textbf{Amazon Web Service}, reason because this are familar to me and already a couple other applications are up and running on the platform. 
The project has been dockerized and contained using docker and uploaded to my git hub. 
From there an \textit{EC2} instance was deployed running a Ubuntu system. 
Docker was then installed on the virtual machine and the project's git repository cloned. 
After that it was just \textit{docker-compose build and up}. \par
An effort was also made to host the application on \textbf{Heroku}, but unlike before I was having a static file problem which I'm currently debugging.

\end{document}

