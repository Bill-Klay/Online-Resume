\documentclass[11pt]{article}
\usepackage{fullpage}
\usepackage{tikz}
\usepackage{graphicx}
\usepackage{hyperref}
\pagenumbering{gobble}
\hypersetup{
    colorlinks=true,
    linkcolor=blue,
    filecolor=red,
	urlcolor=magenta,
}

\begin{document}

\begin{tikzpicture}[remember picture,overlay]
	\node[anchor=north west,yshift=-20pt,xshift=60pt]%
	at (current page.north west)
	{\includegraphics[scale=0.5]{djangoVue.png}};
\end{tikzpicture}

\begin{flushright}
	\Huge{\textbf{Online Resume}} \\
	\Large{Using Django and Vue.js} \\
	\emph{\large{\today}} \\
\end{flushright}

\begin{flushleft}
	\large{\textbf{Prepared by: } Muhammad Bilal Khan} \\
\end{flushleft}

\vspace{-0.8cm}

\begin{center}
	{\rule{475pt}{1pt}}
\end{center}
	
\begin{center}
	Amazon Web Service: \href{http://ec2-52-23-221-163.compute-1.amazonaws.com:5000/}{http://ec2-52-23-221-163.compute-1.amazonaws.com:5000/} \\
	Heroku: \\
\end{center}

\section{Introduction}
This report was establish for the purpose of deployment of an online website implemented using \href{https://www.djangoproject.com/}{Django} as the backend engine and \href{https://vuejs.org/}{Vue.js} as the frontend for manipulating the DOM basically.
The report discusses the layout of the project folder and describes the content held within.
Overviewing and bit about which file implements what and how.
The \textbf{links} stated at the start of the documents links to some of the important aspects of the task.
\textit{Amazon Web Service} holds the code made live running on a virtual machine.
\textit{Heroku} has the django app deployed on a Python droplet.
Please visit my \textit{Git Hub} for viewing and inspecting the complete code.

\section{File Structure}
This 
\end{document}
